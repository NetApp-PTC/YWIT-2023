\chapter{Python Primer}
If you're new to Python, this section will give you a few things you should know
in order to better understand the projects in this guide. This is by no means a
complete or comprehensive look at the Python language. For that, we recommend looking
at the official Python site and reading through the \href{https://docs.python.org/3/tutorial/}{tutorial}
there.

\begin{tcolorbox}
    Note: for the projects being used here, we are using an implementation of
    Python known as \href{https://micropython.org/}{MicroPython}. This version
    is meant to run on microcontrollers with limited resources. It
    also has built into it libraries for dealing with hardware devices that are
    not part of the standard CPython distribution. Therefore, not all Python examples
    you find online will run on your microcontroller and not all projects for a
    microcontroller can be run on your computer. But a lot of the code can be shared
    so the lessons you learn here can apply to other Python projects.
\end{tcolorbox}

Here is a sample of a small Python script. We will disect and explain what each
section does below:

\begin{lstlisting}[language=Python,caption=An example Python script]
def show(message, repeat=1):
    """This function prints the given message to the
    console as smany times as specified in the
    srepeat parameter.
    """

    for iteration in range(0, repeat):
        print(iteration, message)

name = input("What is your name: ")
show(name)
show(name, repeat=3)
\end{lstlisting}

On line 1, we are defining a function named show. This function accepts two parameters,
message and repeat. The message parameter is required and the repeat parameter
is optional with a default value of 1.

Lines 2 through 5 comprise the docstring for the function. This information is meant
for programmers to read and explains what the function does. It does not affect how
the function works.

Line 7 starts a loop. The loop will repeat the statements in the loop body until
a condition is met. In this case, it will loop until it has performed the operation
for each repeat.

Line 8 is the body of the loop. This statement will print the message that the user
passed in to the console along with the iteration number of the loop.

Line 10 prompts the user for their name and saves the result in a variable called
name.

Line 11 calls our show function which will print the user's name once (the default).

Line 12 calls our show function again, this time saying that we want to repeat the
loop of printing the name twice.\newline

Running the program, we will see output like this:
\begin{verbatim}
$ python program.py
What is your name: Emily
0 Emily
0 Emily
1 Emily
2 Emily
$
\end{verbatim}

Another feature that you'll see used often in Python are classes. Classes are a
convienent way to model something in your program that holds state and implements
functionality. For example, let's say that we are writing a game about racing go-
karts. We need to allow each player to have their own kart and keep track of how
fast it is going, which way they are turning, and allow the kart to speed up and
slow down. Here is a small class that will help us do that:

\begin{lstlisting}[language=Python,caption=An example of a Python class]
class Kart:
    MAXIMUM_SPEED = 100

    def __init__(self):
        """The kart starts motionless at the beginning"""
        self._speed = 0
        self._direction = 0
        self._acceleration = 0

    def brake(self):
        """This is called when the user presses the brake button"""
        self._acceleration = -5

    def accelerate(self):
        """This is called when the user presses the accelerator button"""
        self._acceleration = 5

    def steer(self, direction):
        """This is called when the user presses left or right"""
        self._direction = direction

    def update(self, ticks):
        """Update will be called by our game engine and will be
        provided the number of ticks since it was last called.
        """

        self._speed += self._acceleration * ticks

        # limit our speed so that we don't go faster than our
        # kart is allowed to, or slower than 0
        if self._speed > Kart.MAXIMUM_SPEED:
            self._speed = Kart.MAXIMUM_SPEED
        if self._speed < 0:
            self._speed = 0
\end{lstlisting}

Looking at this class, there are 5 methods. The first one (on line 4) is a special
method that is called by the Python interpreter whenever a new Kart is created. It
will initialize some variables for this particular Kart object.

You may have noticed that the first method takes a parameter called "self". This
is the first parameter of all methods in a class in Python. It is automatically passed
by the interpreter and is a reference to the current object. It lets us access the
variables that belong to the class, like those we defined in the \_\_init\_\_ method.

Speaking of the variables in the \_\_init\_\_ method. Notice how we named them all with
an underscore? This is a convention in Python that says they are private to our class
and that code written outside of the class shouldn't access them directly. That means
that our class should provide ways to modify or read these variables via other methods.

The second method starts on line 10. This is called when the player presses the
brake button on their controller and will set our Kart's acceleration to a negative
value so that we start to slow down. It modifies the private \_acceleration member of
the class.

The third starts on line 14. It is the opposite of braking and will start speeding
our Kart up when the user presses the accelerator. It aslo modifies the private
\_acceleration member of the class.

The fourth method, line 18, is again something to deal with user input. This time
we can see that it takes a second parameter, direction. If the user presses left on
their controller, then we can expect left to be passed here. The same for right. We
will modify the private \_direction member here.

Finally, we have a fifth method starting on line 22. This method is called by our
game engine and uses the class members to determine what happens to the Kart throughout
the game. That is, it is asking the Kart to update itself at a certain moment in time
(usually once per frame) so that next time it draws it to the screen, it will be
in the updated location.

Notice in the last method, we are accessing not only our own variables, \_speed, and
\_acceleration, but we are also reading a class variable, Kart.MAXIMUM\_SPEED. Unlike
our member variables, a class variable is the same for all instances of a class. It
is useful here to keep the game fair so that all Karts have the same limitation on
their speed.